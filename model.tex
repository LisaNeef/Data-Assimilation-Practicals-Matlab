The 3-variable model of \citet{Lorenz1963} is frequently used to test data assimilation algorithms because, like the real atmosphere, it displays deterministic chaos. 
\begin{eqnarray}
\frac{dx}{dt} &=& f_1(x,y) = \sigma (y-x) \label{eq:x} \\
\frac{dy}{dt} &=& f_2(x,y,z) = \rho x - y - xz \label{eq:y} \\
\frac{dz}{dt} &=& f_3(x,y,z) = xy-\beta z. \label{eq:z} 
\end{eqnarray}
%
We will use this model in a set of data assimilation experiments, with the following parameters:\\
\centerline{
\begin{tabular}{ll}
	$\sigma = 10$ 	& --the Prandtl number \\
	$\rho = 28$ 	& --the normalized Rayleigh number \\
	$\beta = 8/3$ 	& --a nondimensional wavenumber.
\end{tabular}
}
\\
The Matlab program \texttt{EnKF\_l63.m} runs the Lorenz model forward given a set of initial conditions, along with a forecast ensemble. 
It also runs an optional ensemble Kalman filter (EnKF) that assimilates regularly-spaced observations of some or all the model variables, but first we'll focus on the model.
%
The model and assimilation inputs are set in \texttt{set\_enkf\_inputs\_template.m}.
Copy this file to a new file called \texttt{set\_enkf\_inputs.m},
then open this file in an editor and set \texttt{run\_filter} to zero.
Now 
\begin{verbatim}
> E = set_enkf_inputs
\end{verbatim}
sets the input variables and parameters. 
The output, $E$, is a matlab structure that contains all the model and assimilation parameters needed to run the EnKF. 
Now to run the model, type
\begin{verbatim}
> EnKF_l63(E)
\end{verbatim}
Try runnning the model with a few different initial conditions (by re-running \texttt{set\_enkf\_inputs.m}) to get an idea of the overall behavior of the model. 
You'll see that the Lorenz model, like the weather, is chaotic -- small changes in the initial conditions become huge changes down the line. 

You'll see that the ensemble (gray) eventually separates so much from the truth that its mean (green) no longer looks like a typical state of the Lorenz model (i.e. it's no longer on the "strange attractor" of the model). 
If the Lorenz model was a model of the weather, taking the ensemble mean as a weather forecast would be a pretty bad idea -- not only would it be far from the truth, it wouldn't even be physical. 
Fortunately, in the real world we have observations of the weather to correct our forecast models and bring them closer to the truth. 
This will be illustrated in the next section. 
